%\renewcommand{\abstractnamefont}{\normalfont\LARGE\bfseries}
%\renewcommand{\abstracttextfont}{\normalfont\Huge}

\begin{abstract}
\begin{otherlanguage}{french}
\hskip7mm
\begin{spacing}{1.3}
\noindent\rule[2pt]{\textwidth}{2pt}\\[1.2cm]
\scalefont{1.1}Afin de réduire le délai de commercialisation et produire des logiciels de qualité, Mobelite choisit d'adopter les pratiques DevOps dans les projets de ses clients.\\[0.3cm]
Dans ce contexte, ce projet vise à améliorer et optimiser une solution hybride qui combiner les ressources locales de l'entreprise avec les ressources du cloud pour créer des pipelines de livraison continue qui assurent la gestion des changements, depuis le code source jusqu'à la mise en production. Il intègre l'outil Ansible pour la gestion de la configuration et l'automatisation du déploiement.\\[0.3cm]
 Aussi, il insère un système de recovery et de failover à base de Kubernetes et il utilise des services AWS comme Elastic Container Registry pour le repository. Toutes les images Docker sont déployées dans des conteneurs.\\[0.3cm]
Après avoir effectué une étude comparative des solutions existantes, identifié les besoins et conçu l'architecture du projet, il est nécessaire de mettre en place un pipeline de livraison continue optimisé.\\[0.3cm]
Ce projet a permis de minimiser le temps de déploiement dans les différents environnements et d'avoir une résilience contre les pannes d'infrastructures.\\[1.2cm]

\noindent\rule[2pt]{\textwidth}{2pt}
\vspace{1cm}
\textbf{\Large Mots clés:}  
\scalefont{1.1}DevOps, Kubernetes, AWS, Ansible, Pipline, Monitoring,cloud hybride.\\[1.2cm]
\end{spacing}
\end{otherlanguage}
\end{abstract}
\begin{spacing}{1.3}

\textbf{\centering\Large Abstract}\\
\noindent\rule[2pt]{\textwidth}{2pt}\\[1.2cm]
\scalefont{1.1}To reduce time-to-market and produce quality software, Mobelite has chosen to adopt DevOps practices in its clients projects.\\[0.3cm]
In this context, this project aims to improve and optimize a Hybrid cloud solution that combine the local ressources and the cloud for continuous delivery pipeline that handles changes from source code to production. It integrates the Ansible tool for configuration management and deployment automation. Additionally, it incorporates a recovery and failover system based on Kubernetes and utilizes AWS services like Elastic Container Registry for storing all Docker images in containers.\\[0.3cm]
After conducting a comparative study of existing solutions, identifying the needs, and designing the project's architecture, it became necessary to implement an optimized continuous delivery pipeline.\\[0.3cm]
This project has minimized deployment time in different environments and provided resilience against infrastructure failures.\\[1.3cm]

\noindent\rule[2pt]{\textwidth}{2pt}\\[1.2cm]
\textbf{\Large Keywords:}
\scalefont{1.1}DevOps, Kubernetes, AWS, Ansible, Pipeline, Monitoring,Hybrid cloud.\\[0.3cm]
\end{spacing}
