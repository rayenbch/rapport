\chapter*{Introduction générale}
\addcontentsline{toc}{chapter}{Introduction générale}
\markboth{Introduction générale}{Introduction générale}
\label{chap:introduction}
\textsf{\fontfamily{qtm}\selectfont\scalefont{1.5}La montée en puissance de l'entreprise numérique nécessite une profonde révision des méthodes de développement d'applications. Il n'est plus viable d'attendre six mois pour obtenir les livrables de développement. Avec les exigences de rapidité sur le marché et l'évolution des projets dans des configurations logicielles et d'infrastructure de plus en plus complexes, qui entraînent des risques opérationnels et de planification, il est devenu essentiel d'industrialiser les tests et de simplifier les déploiements en production. En rapprochant les équipes de développement, de test et d'exploitation, le DevOps répond précisément à ce défi numérique. Les entreprises en sont maintenant pleinement conscientes.\\[1cm]
Parmi les pratiques couramment utilisées en DevOps, on retrouve la livraison continue et la gestion de la configuration. La livraison continue est une approche logicielle qui permet aux organisations de fournir rapidement et efficacement de nouvelles fonctionnalités aux utilisateurs. L'idée fondamentale de la livraison continue est de créer un processus reproductible et fiable d'amélioration progressive pour faire passer le logiciel du concept au client. L'objectif de la livraison continue est de permettre un flux constant de changements vers la production en utilisant un pipeline automatisé de logiciels. C'est ce pipeline de livraison continue qui rend tout cela possible.\\[1cm]
Ce projet s'inscrit dans cette perspective en cherchant à  créer un espace partagé entre une infrastructure Cloud et un Datacenter local de livraison continue en intégrant Ansible ,Kubernetes et des services AWS pour créer des containers pour le monitoring. Mobelite a entrepris ce projet afin de résoudre plusieurs problématiques.\\[1cm]
Ce rapport présente les principales réalisations effectuées au cours de ce projet et est structuré en quatre chapitres :\\[1cm]
Dans le premier chapitre,nous présentons l’entreprise d’accueil Mobelite , après nous explorons la problématique de notre sujet.Puis les notions de base comme microservices ,DevOps et cloud .Ensuite, nous avons définir la méthodologie de développement .Enfin, nous avons présenter la planifications des sprints.\\[1cm]
Dans la deuxième chapitre, nous présentons une étude comparative des solutions existantes  les
plus répandus sur le marché et le plus adéquate, ainsi que l’outil choisi dans chaque solution.\\[1cm]
Dans la troisième chapitre, analyse et conception,  nous exposerons les besoins
fonctionnels et non fonctionnels, les acteurs, les cas d’utilisations et l’architecture physique global .\\[1cm]
Dans Le dernier chapitre est consacré pour présenter les tâches réalisées pour implémenter notre
projet.
}